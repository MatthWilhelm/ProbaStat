%% Generated by Sphinx.
\def\sphinxdocclass{jupyterBook}
\documentclass[letterpaper,10pt,english]{jupyterBook}
\ifdefined\pdfpxdimen
   \let\sphinxpxdimen\pdfpxdimen\else\newdimen\sphinxpxdimen
\fi \sphinxpxdimen=.75bp\relax
\ifdefined\pdfimageresolution
    \pdfimageresolution= \numexpr \dimexpr1in\relax/\sphinxpxdimen\relax
\fi
%% let collapsible pdf bookmarks panel have high depth per default
\PassOptionsToPackage{bookmarksdepth=5}{hyperref}
%% turn off hyperref patch of \index as sphinx.xdy xindy module takes care of
%% suitable \hyperpage mark-up, working around hyperref-xindy incompatibility
\PassOptionsToPackage{hyperindex=false}{hyperref}
%% memoir class requires extra handling
\makeatletter\@ifclassloaded{memoir}
{\ifdefined\memhyperindexfalse\memhyperindexfalse\fi}{}\makeatother

\PassOptionsToPackage{warn}{textcomp}

\catcode`^^^^00a0\active\protected\def^^^^00a0{\leavevmode\nobreak\ }
\usepackage{cmap}
\usepackage{fontspec}
\defaultfontfeatures[\rmfamily,\sffamily,\ttfamily]{}
\usepackage{amsmath,amssymb,amstext}
\usepackage{polyglossia}
\setmainlanguage{english}



\setmainfont{FreeSerif}[
  Extension      = .otf,
  UprightFont    = *,
  ItalicFont     = *Italic,
  BoldFont       = *Bold,
  BoldItalicFont = *BoldItalic
]
\setsansfont{FreeSans}[
  Extension      = .otf,
  UprightFont    = *,
  ItalicFont     = *Oblique,
  BoldFont       = *Bold,
  BoldItalicFont = *BoldOblique,
]
\setmonofont{FreeMono}[
  Extension      = .otf,
  UprightFont    = *,
  ItalicFont     = *Oblique,
  BoldFont       = *Bold,
  BoldItalicFont = *BoldOblique,
]



\usepackage[Bjarne]{fncychap}
\usepackage[,numfigreset=1,mathnumfig]{sphinx}

\fvset{fontsize=\small}
\usepackage{geometry}


% Include hyperref last.
\usepackage{hyperref}
% Fix anchor placement for figures with captions.
\usepackage{hypcap}% it must be loaded after hyperref.
% Set up styles of URL: it should be placed after hyperref.
\urlstyle{same}

\addto\captionsenglish{\renewcommand{\contentsname}{À propos du cours}}

\usepackage{sphinxmessages}



        % Start of preamble defined in sphinx-jupyterbook-latex %
         \usepackage[Latin,Greek]{ucharclasses}
        \usepackage{unicode-math}
        % fixing title of the toc
        \addto\captionsenglish{\renewcommand{\contentsname}{Contents}}
        \hypersetup{
            pdfencoding=auto,
            psdextra
        }
        % End of preamble defined in sphinx-jupyterbook-latex %
        

\title{My sample book}
\date{Aug 18, 2022}
\release{}
\author{The Jupyter Book Community}
\newcommand{\sphinxlogo}{\vbox{}}
\renewcommand{\releasename}{}
\makeindex
\begin{document}

\pagestyle{empty}
\sphinxmaketitle
\pagestyle{plain}
\sphinxtableofcontents
\pagestyle{normal}
\phantomsection\label{\detokenize{intro::doc}}


\sphinxAtStartPar
This is a small sample book to give you a feel for how book content is
structured.
It shows off a few of the major file types, as well as some sample content.
It does not go in\sphinxhyphen{}depth into any particular topic \sphinxhyphen{} check out \sphinxhref{https://jupyterbook.org}{the Jupyter Book documentation} for more information.

\sphinxAtStartPar
Check out the content pages bundled with this sample book to see more.
\begin{itemize}
\item {} 
\sphinxAtStartPar
À propos du cours

\begin{itemize}
\item {} 
\sphinxAtStartPar
{\hyperref[\detokenize{Introduction::doc}]{\sphinxcrossref{Introduction}}}

\end{itemize}
\end{itemize}
\begin{itemize}
\item {} 
\sphinxAtStartPar
Statistique Exploratoire

\begin{itemize}
\item {} 
\sphinxAtStartPar
{\hyperref[\detokenize{Statistiques_exploratoire/types_de_donn_xe9es::doc}]{\sphinxcrossref{Types de données}}}

\item {} 
\sphinxAtStartPar
{\hyperref[\detokenize{Statistiques_exploratoire/graphiques::doc}]{\sphinxcrossref{Graphiques}}}

\item {} 
\sphinxAtStartPar
{\hyperref[\detokenize{Statistiques_exploratoire/synth_xe8ses_num_xe9riques::doc}]{\sphinxcrossref{Synthèses numériques}}}

\item {} 
\sphinxAtStartPar
{\hyperref[\detokenize{Statistiques_exploratoire/boxplot::doc}]{\sphinxcrossref{Boxplot}}}

\item {} 
\sphinxAtStartPar
{\hyperref[\detokenize{Statistiques_exploratoire/loi_normale::doc}]{\sphinxcrossref{Loi normale}}}

\item {} 
\sphinxAtStartPar
{\hyperref[\detokenize{Statistiques_exploratoire/quelques_principes_g_xe9n_xe9raux::doc}]{\sphinxcrossref{Quelques principes généraux}}}

\end{itemize}
\end{itemize}
\begin{itemize}
\item {} 
\sphinxAtStartPar
Probabilité

\begin{itemize}
\item {} 
\sphinxAtStartPar
{\hyperref[\detokenize{Probabilit_xe9/concepts_de_base::doc}]{\sphinxcrossref{Concepts de base}}}

\item {} 
\sphinxAtStartPar
{\hyperref[\detokenize{Probabilit_xe9/arrangements_et_combinaisons::doc}]{\sphinxcrossref{Arrangements et combinaisons}}}

\item {} 
\sphinxAtStartPar
{\hyperref[\detokenize{Probabilit_xe9/probabilite_conditionnelle_independance::doc}]{\sphinxcrossref{Probabilité conditionelle et indépendence}}}

\item {} 
\sphinxAtStartPar
{\hyperref[\detokenize{Probabilit_xe9/probabilites_totales_et_theoreme_de_bayes::doc}]{\sphinxcrossref{Probabilités totales et théorème de Bayes}}}

\item {} 
\sphinxAtStartPar
{\hyperref[\detokenize{Probabilit_xe9/variables_aleatoires_discretes::doc}]{\sphinxcrossref{Variables aléatoires discrètes}}}

\item {} 
\sphinxAtStartPar
{\hyperref[\detokenize{Probabilit_xe9/variables_aleatoires_continues::doc}]{\sphinxcrossref{Variables aléatoires continues}}}

\item {} 
\sphinxAtStartPar
{\hyperref[\detokenize{Probabilit_xe9/variables_aleatoires_conjointes::doc}]{\sphinxcrossref{Variables aléatoires conjointes}}}

\item {} 
\sphinxAtStartPar
{\hyperref[\detokenize{Probabilit_xe9/valeurs_caracteristiques::doc}]{\sphinxcrossref{Valeurs caractéristiques}}}

\item {} 
\sphinxAtStartPar
{\hyperref[\detokenize{Probabilit_xe9/theoreme_fondamentaux::doc}]{\sphinxcrossref{Théorème fondamentaux}}}

\item {} 
\sphinxAtStartPar
{\hyperref[\detokenize{Probabilit_xe9/test::doc}]{\sphinxcrossref{Test live code}}}

\end{itemize}
\end{itemize}
\begin{itemize}
\item {} 
\sphinxAtStartPar
Statistique inférentielle

\begin{itemize}
\item {} 
\sphinxAtStartPar
{\hyperref[\detokenize{Statistique_inf_xe9rentielle/introduction::doc}]{\sphinxcrossref{Introduction}}}

\item {} 
\sphinxAtStartPar
{\hyperref[\detokenize{Statistique_inf_xe9rentielle/estimation_de_parametres::doc}]{\sphinxcrossref{Estimation de paramètres}}}

\item {} 
\sphinxAtStartPar
{\hyperref[\detokenize{Statistique_inf_xe9rentielle/proprietes_estimateur::doc}]{\sphinxcrossref{Propriétés d’un estimateur}}}

\item {} 
\sphinxAtStartPar
{\hyperref[\detokenize{Statistique_inf_xe9rentielle/estimation_par_intervalle::doc}]{\sphinxcrossref{Estimation par intervalle}}}

\item {} 
\sphinxAtStartPar
{\hyperref[\detokenize{Statistique_inf_xe9rentielle/tests_hypotheses_statistiques::doc}]{\sphinxcrossref{Tests d’hypothèses statistiques}}}

\item {} 
\sphinxAtStartPar
{\hyperref[\detokenize{Statistique_inf_xe9rentielle/tests_et_ic::doc}]{\sphinxcrossref{Tests et IC}}}

\item {} 
\sphinxAtStartPar
{\hyperref[\detokenize{Statistique_inf_xe9rentielle/test_chi2::doc}]{\sphinxcrossref{Test du chi2}}}

\item {} 
\sphinxAtStartPar
{\hyperref[\detokenize{Statistique_inf_xe9rentielle/puissance_test::doc}]{\sphinxcrossref{Puissance d’un test}}}

\end{itemize}
\end{itemize}
\begin{itemize}
\item {} 
\sphinxAtStartPar
Régréssion linéaire

\begin{itemize}
\item {} 
\sphinxAtStartPar
{\hyperref[\detokenize{R_xe9gr_xe9ssion_lin_xe9aire/introduction::doc}]{\sphinxcrossref{Introduction}}}

\item {} 
\sphinxAtStartPar
{\hyperref[\detokenize{R_xe9gr_xe9ssion_lin_xe9aire/cas_general::doc}]{\sphinxcrossref{Régréssion linéaire: cas général}}}

\item {} 
\sphinxAtStartPar
{\hyperref[\detokenize{R_xe9gr_xe9ssion_lin_xe9aire/tests::doc}]{\sphinxcrossref{Régréssion linéaire: tests}}}

\item {} 
\sphinxAtStartPar
{\hyperref[\detokenize{R_xe9gr_xe9ssion_lin_xe9aire/hypotheses_et_diagnostics::doc}]{\sphinxcrossref{Régréssion linéaire: hypothèses et diagnostics}}}

\end{itemize}
\end{itemize}

\sphinxstepscope


\part{À propos du cours}

\sphinxstepscope


\chapter{Introduction}
\label{\detokenize{Introduction:introduction}}\label{\detokenize{Introduction::doc}}

\section{Statistiques: une définition ?}
\label{\detokenize{Introduction:statistiques-une-definition}}
\sphinxAtStartPar
Les mathématiques (du grec \sphinxstyleemphasis{Mathema} \(\approx\) apprendre) sont une manière:
\begin{enumerate}
\sphinxsetlistlabels{\arabic}{enumi}{enumii}{}{.}%
\item {} 
\sphinxAtStartPar
d’exprimer une grande variété de notions complexes avec précision et cohérence.

\item {} 
\sphinxAtStartPar
de « \sphinxstyleemphasis{légitimer les conquêtes de notre intuition} » (Jacques Hadamard)
%
\begin{footnote}[1]\sphinxAtStartFootnote
La citation exacte est \sphinxhref{http://idm-old.math.cnrs.fr/Jacques-Hadamard-passeur.html}{« La rigueur n’a d’autre objet que de sanctionner et de légitimer les conquêtes de l’intuition »}.
%
\end{footnote}
: raisonner rigoureusement à partir d’hypothèses, tirer les conclusions correctes d’une observation, d’une expérience, etc.

\end{enumerate}

\sphinxAtStartPar
La statistique souffre d’un problème de nomenclature. On a deux « statistiques » (définitions \sphinxhref{http://stella.atilf.fr/Dendien/scripts/tlfiv5/visusel.exe?12\%3Bs=467953305\%3Br=1\%3Bnat=\%3Bsol=1\%3B}{TLF}):
\begin{enumerate}
\sphinxsetlistlabels{\arabic}{enumi}{enumii}{}{.}%
\item {} 
\sphinxAtStartPar
« Recueil de données numériques concernant des faits économiques et sociaux »; par exemple des données démographiques (répartition en âge, métiers, etc. de la population), ou économiques (taux de chômage, salaire médian, etc.).«L’ensemble des connaissances que doit posséder un homme d’État », introduit en allemand \sphinxstyleemphasis{Statistik} par l’économiste G. Achenwall (1719\sphinxhyphen{}1772) (de l’italien \sphinxstyleemphasis{statista}, homme d’’Etat).

\item {} 
\sphinxAtStartPar
« Branche des mathématiques ayant pour objet l’analyse et l’interprétation de données quantifiables.»

\end{enumerate}
\begin{gather*}
        \color{red}{Utiliser les maths} \\ \text{pour} \\ \color{red}{extraire des informations} \\ \text{à partir de}\\ \color{red}{données}\\ \text{en présence}\\ \color{red}{d'incertitudes}
\end{gather*}
\sphinxAtStartPar
Les données sont absolument partout de nos jours.
\begin{enumerate}
\sphinxsetlistlabels{\arabic}{enumi}{enumii}{}{.}%
\item {} 
\sphinxAtStartPar
OFS: démographie, chômage \(\rightarrow\) décisions politiques;

\item {} 
\sphinxAtStartPar
science: expérience \(\rightarrow\) données \(\rightarrow\) conclusion;

\item {} 
\sphinxAtStartPar
utilisation d’internet: « cookies » \(\rightarrow\) publicités ciblées.

\end{enumerate}


\section{Et les probabilités ?}
\label{\detokenize{Introduction:et-les-probabilites}}
\sphinxAtStartPar
Les probabilités nous aident à appréhender \sphinxstylestrong{l’incertitude}; elles permettent de la transcrire en un formalisme mathématique.
\begin{enumerate}
\sphinxsetlistlabels{\arabic}{enumi}{enumii}{}{.}%
\item {} 
\sphinxAtStartPar
C’est la discipline qui étudie les phénomènes aléatoires (ou \sphinxstyleemphasis{stochastiques}).

\item {} 
\sphinxAtStartPar
C’est la base indispensable à toute étude mathématiquement rigoureuse de ces phénomènes.

\end{enumerate}

\sphinxAtStartPar
Les probabilités nous donnent \sphinxstylestrong{le formalisme} dans lequel on peut comprendre et quantifier l’effet que la présence d’incertitude dans les données a sur notre analyse de ces données.


\section{Le but de la statistique}
\label{\detokenize{Introduction:le-but-de-la-statistique}}
\sphinxAtStartPar
L’expérience montre que de nombreuses expériences sont \sphinxstyleemphasis{intrinsèquement} aléatoire: jet de dé, un tirage au sort, une campagne de vaccination.D’autres ne le sont peut\sphinxhyphen{}être pas intrinsèquement, mais il s’avère impossible de les reproduire exactement: tir au panier, mesure physique etc. Ainsi, \sphinxstyleemphasis{le hasard est une composante essentielle de bien des expériences}.La plupart du temps, le but de la statistique est de \sphinxstylestrong{comprendre ce hasard}.

\sphinxAtStartPar
On peut identifier quatres étapes majeures de la démarche statistique:
\begin{enumerate}
\sphinxsetlistlabels{\arabic}{enumi}{enumii}{}{.}%
\item {} 
\sphinxAtStartPar
planification de l’expérience; (développement théorique du problème, élaboration du plan expérimental);

\item {} 
\sphinxAtStartPar
collecte des données;

\item {} 
\sphinxAtStartPar
\sphinxstylestrong{analyse des données};

\item {} 
\sphinxAtStartPar
présentation des résultats et conclusions / actions;

\end{enumerate}

\sphinxAtStartPar
Ce cours se concentre sur \sphinxstylestrong{l’analyse des données}.
Je conseille fortement la référence \sphinxhref{https://swisscovery.slsp.ch/permalink/41SLSP\_NETWORK/1ufb5t2/alma991170199983205501}{suivante}: Cox, D. R.  and Donnelly, C. A. (2011) \sphinxstyleemphasis{Principles of applied statistics}, Cambridge, UK: Cambridge University Press.
\begin{itemize}
\item {} 
\sphinxAtStartPar
\sphinxstylestrong{L’analyse exploratoire des données}: consiste en l’utilisation de méthodes simples, intuitives, essentiellement graphiques. Son objectif est l’identification informelle de la structure d’un jeu de données (tendances, formes, observation atypiques). Elle permet donc de se familiariser avec les données.

\end{itemize}

\sphinxAtStartPar
L’analyse exploratoire suggère des hypothèses de travail et des modèles, qui sont formalisés et vérifiés dans le second pôle:
\begin{itemize}
\item {} 
\sphinxAtStartPar
\sphinxstylestrong{L’analyse confirmatoire des données}: elle conduit à des conclusions statistiques à partir de données en utilisant des notions de la théorie des probabilités. Cette partie plus formelle concerne notamment des méthodes de test, d’estimation et de prévision.

\end{itemize}

\sphinxAtStartPar
On distingue en général deux grands types d’études: expérimentales et observationnelles. La démarche est fondamentalement différente, ainsi que les conclusions que l’on peut en tirer.


\begin{savenotes}\sphinxattablestart
\centering
\begin{tabulary}{\linewidth}[t]{|T|T|T|}
\hline

\sphinxAtStartPar

&\sphinxstyletheadfamily 
\sphinxAtStartPar
Etude expérimentale
&\sphinxstyletheadfamily 
\sphinxAtStartPar
Etude  observationnelle
\\
\hline
\sphinxAtStartPar
Situation
&
\sphinxAtStartPar
Sous contrôle: les paramètres que vous souhaitez sont ceux que vous obtenez
&
\sphinxAtStartPar
Donnée : ce que vous observez est tout ce que vous avez
\\
\hline
\sphinxAtStartPar
Analyse
&
\sphinxAtStartPar
Aisée et planifiée
&
\sphinxAtStartPar
Potentiellement difficile
\\
\hline
\sphinxAtStartPar
Interprétattion
&
\sphinxAtStartPar
causale (si fait correctement)
&
\sphinxAtStartPar
associative, uniquemen
\\
\hline
\end{tabulary}
\par
\sphinxattableend\end{savenotes}
\begin{itemize}
\item {} 
\sphinxAtStartPar
\sphinxstylestrong{L’analyse exploratoire des données}: consiste en l’utilisation de méthodes simples, intuitives, essentiellement graphiques. Son objectif est l’identification informelle de la structure d’un jeu de données (tendances, formes, observation atypiques). Elle permet donc de se familiariser avec les données.

\end{itemize}

\sphinxAtStartPar
L’analyse exploratoire suggère des hypothèses de travail et des modèles, qui sont formalisés et vérifiés dans le second pôle:
\begin{itemize}
\item {} 
\sphinxAtStartPar
\sphinxstylestrong{L’analyse confirmatoire des données}: elle conduit à des conclusions statistiques à partir de données en utilisant des notions de la théorie des probabilités. Cette partie plus formelle concerne notamment des méthodes de test, d’estimation et de prévision.

\end{itemize}

\sphinxAtStartPar
On distingue en général deux grands types d’études: expérimentales et observationnelles. La démarche est fondamentalement différente, ainsi que les conclusions que l’on peut en tirer.

\begin{figure}[htbp]
\centering
\capstart

\noindent\sphinxincludegraphics[height=350\sphinxpxdimen]{{table}.pdf}
\caption{Here is my figure caption!}\label{\detokenize{Introduction:directive-fi}}\end{figure}


\bigskip\hrule\bigskip


\sphinxstepscope


\part{Statistique Exploratoire}

\sphinxstepscope


\chapter{Types de données}
\label{\detokenize{Statistiques_exploratoire/types_de_donn_xe9es:types-de-donnees}}\label{\detokenize{Statistiques_exploratoire/types_de_donn_xe9es::doc}}
\sphinxstepscope


\chapter{Graphiques}
\label{\detokenize{Statistiques_exploratoire/graphiques:graphiques}}\label{\detokenize{Statistiques_exploratoire/graphiques::doc}}
\sphinxstepscope


\chapter{Synthèses numériques}
\label{\detokenize{Statistiques_exploratoire/synth_xe8ses_num_xe9riques:syntheses-numeriques}}\label{\detokenize{Statistiques_exploratoire/synth_xe8ses_num_xe9riques::doc}}
\sphinxstepscope


\chapter{Boxplot}
\label{\detokenize{Statistiques_exploratoire/boxplot:boxplot}}\label{\detokenize{Statistiques_exploratoire/boxplot::doc}}
\sphinxstepscope


\chapter{Loi normale}
\label{\detokenize{Statistiques_exploratoire/loi_normale:loi-normale}}\label{\detokenize{Statistiques_exploratoire/loi_normale::doc}}
\sphinxstepscope


\chapter{Quelques principes généraux}
\label{\detokenize{Statistiques_exploratoire/quelques_principes_g_xe9n_xe9raux:quelques-principes-generaux}}\label{\detokenize{Statistiques_exploratoire/quelques_principes_g_xe9n_xe9raux::doc}}
\sphinxstepscope


\part{Probabilité}

\sphinxstepscope


\chapter{Concepts de base}
\label{\detokenize{Probabilit_xe9/concepts_de_base:concepts-de-base}}\label{\detokenize{Probabilit_xe9/concepts_de_base::doc}}
\sphinxAtStartPar
Some \sphinxstylestrong{intro Markdown}!

\begin{sphinxuseclass}{cell}
\begin{sphinxuseclass}{tag_mytag}\begin{sphinxVerbatimInput}

\begin{sphinxuseclass}{cell_input}
\begin{sphinxVerbatim}[commandchars=\\\{\}]
\PYG{n+nb}{print}\PYG{p}{(}\PYG{l+s+s2}{\PYGZdq{}}\PYG{l+s+s2}{A python cell}\PYG{l+s+s2}{\PYGZdq{}}\PYG{p}{)}
\end{sphinxVerbatim}

\end{sphinxuseclass}\end{sphinxVerbatimInput}
\begin{sphinxVerbatimOutput}

\begin{sphinxuseclass}{cell_output}
\begin{sphinxVerbatim}[commandchars=\\\{\}]
A python cell
\end{sphinxVerbatim}

\end{sphinxuseclass}\end{sphinxVerbatimOutput}

\end{sphinxuseclass}
\end{sphinxuseclass}

\section{A section}
\label{\detokenize{Probabilit_xe9/concepts_de_base:a-section}}
\begin{figure}[htbp]
\centering
\capstart

\noindent\sphinxincludegraphics[height=350\sphinxpxdimen]{{Probabilité/tikz}.#}
\caption{Here is my figure caption!}\label{\detokenize{Probabilit_xe9/concepts_de_base:directive-fig}}\end{figure}

\sphinxAtStartPar
And some more Markdown…


\section{A section}
\label{\detokenize{Probabilit_xe9/concepts_de_base:id1}}
\sphinxstepscope


\chapter{Arrangements et combinaisons}
\label{\detokenize{Probabilit_xe9/arrangements_et_combinaisons:arrangements-et-combinaisons}}\label{\detokenize{Probabilit_xe9/arrangements_et_combinaisons::doc}}

\section{Section 1}
\label{\detokenize{Probabilit_xe9/arrangements_et_combinaisons:section-1}}\label{Probabilité/arrangements_et_combinaisons:theorem-0}
\begin{sphinxadmonition}{note}{Theorem 1 (Théorème Centrale Limite )}



\sphinxAtStartPar
Soient \(X_1, \dots, X_n, \dots\) une suite de variables aléatoires réelles indépendantes et identiquement distribuées de moyennes et de variances communes \(\mu\) et \(\sigma^2\) respectivement. Alors, pour \(n \rightarrow \infty\), \(Z_n\) est approximativement Gaussienne, c’est\sphinxhyphen{}à\sphinxhyphen{}dire qu’on a:
\label{equation:Probabilité/arrangements_et_combinaisons:2a996ed5-f4cc-46fd-8896-11853fb30128}\begin{equation}
\mathbb{P}( Z_n \leq z ) \stackrel{n\rightarrow\infty}{\longrightarrow} \Phi(z).
\end{equation}
\sphinxAtStartPar
Donc, pour \(n\) suffisamment grand, \(\overline{X}_n \sim \mathcal{N}(\mu, \sigma^2)\).
\end{sphinxadmonition}


\subsection{Sub\sphinxhyphen{}Section 1.1}
\label{\detokenize{Probabilit_xe9/arrangements_et_combinaisons:sub-section-1-1}}

\subsection{Sub\sphinxhyphen{}Section 1.2}
\label{\detokenize{Probabilit_xe9/arrangements_et_combinaisons:sub-section-1-2}}

\section{Section 2}
\label{\detokenize{Probabilit_xe9/arrangements_et_combinaisons:section-2}}
\sphinxstepscope


\chapter{Probabilité conditionelle et indépendence}
\label{\detokenize{Probabilit_xe9/probabilite_conditionnelle_independance:probabilite-conditionelle-et-independence}}\label{\detokenize{Probabilit_xe9/probabilite_conditionnelle_independance::doc}}
\sphinxstepscope


\chapter{Probabilités totales et théorème de Bayes}
\label{\detokenize{Probabilit_xe9/probabilites_totales_et_theoreme_de_bayes:probabilites-totales-et-theoreme-de-bayes}}\label{\detokenize{Probabilit_xe9/probabilites_totales_et_theoreme_de_bayes::doc}}
\sphinxstepscope


\chapter{Variables aléatoires discrètes}
\label{\detokenize{Probabilit_xe9/variables_aleatoires_discretes:variables-aleatoires-discretes}}\label{\detokenize{Probabilit_xe9/variables_aleatoires_discretes::doc}}
\sphinxstepscope


\chapter{Variables aléatoires continues}
\label{\detokenize{Probabilit_xe9/variables_aleatoires_continues:variables-aleatoires-continues}}\label{\detokenize{Probabilit_xe9/variables_aleatoires_continues::doc}}
\sphinxstepscope


\chapter{Variables aléatoires conjointes}
\label{\detokenize{Probabilit_xe9/variables_aleatoires_conjointes:variables-aleatoires-conjointes}}\label{\detokenize{Probabilit_xe9/variables_aleatoires_conjointes::doc}}
\sphinxstepscope


\chapter{Valeurs caractéristiques}
\label{\detokenize{Probabilit_xe9/valeurs_caracteristiques:valeurs-caracteristiques}}\label{\detokenize{Probabilit_xe9/valeurs_caracteristiques::doc}}
\sphinxstepscope


\chapter{Théorème fondamentaux}
\label{\detokenize{Probabilit_xe9/theoreme_fondamentaux:theoreme-fondamentaux}}\label{\detokenize{Probabilit_xe9/theoreme_fondamentaux::doc}}
\sphinxstepscope


\chapter{Test live code}
\label{\detokenize{Probabilit_xe9/test:test-live-code}}\label{\detokenize{Probabilit_xe9/test::doc}}
\begin{sphinxuseclass}{cell}\begin{sphinxVerbatimInput}

\begin{sphinxuseclass}{cell_input}
\begin{sphinxVerbatim}[commandchars=\\\{\}]
\PYG{k+kn}{import} \PYG{n+nn}{plotly}\PYG{n+nn}{.}\PYG{n+nn}{graph\PYGZus{}objects} \PYG{k}{as} \PYG{n+nn}{go}
\PYG{k+kn}{from} \PYG{n+nn}{plotly}\PYG{n+nn}{.}\PYG{n+nn}{subplots} \PYG{k+kn}{import} \PYG{n}{make\PYGZus{}subplots}
\PYG{k+kn}{import} \PYG{n+nn}{numpy} \PYG{k}{as} \PYG{n+nn}{np}

\PYG{n}{z} \PYG{o}{=}   \PYG{p}{[}\PYG{p}{[}\PYG{l+m+mi}{2}\PYG{p}{,} \PYG{l+m+mi}{4}\PYG{p}{,} \PYG{l+m+mi}{7}\PYG{p}{,} \PYG{l+m+mi}{12}\PYG{p}{,} \PYG{l+m+mi}{13}\PYG{p}{,} \PYG{l+m+mi}{14}\PYG{p}{,} \PYG{l+m+mi}{15}\PYG{p}{,} \PYG{l+m+mi}{16}\PYG{p}{]}\PYG{p}{,}
       \PYG{p}{[}\PYG{l+m+mi}{3}\PYG{p}{,} \PYG{l+m+mi}{1}\PYG{p}{,} \PYG{l+m+mi}{6}\PYG{p}{,} \PYG{l+m+mi}{11}\PYG{p}{,} \PYG{l+m+mi}{12}\PYG{p}{,} \PYG{l+m+mi}{13}\PYG{p}{,} \PYG{l+m+mi}{16}\PYG{p}{,} \PYG{l+m+mi}{17}\PYG{p}{]}\PYG{p}{,}
       \PYG{p}{[}\PYG{l+m+mi}{4}\PYG{p}{,} \PYG{l+m+mi}{2}\PYG{p}{,} \PYG{l+m+mi}{7}\PYG{p}{,} \PYG{l+m+mi}{7}\PYG{p}{,} \PYG{l+m+mi}{11}\PYG{p}{,} \PYG{l+m+mi}{14}\PYG{p}{,} \PYG{l+m+mi}{17}\PYG{p}{,} \PYG{l+m+mi}{18}\PYG{p}{]}\PYG{p}{,}
       \PYG{p}{[}\PYG{l+m+mi}{5}\PYG{p}{,} \PYG{l+m+mi}{3}\PYG{p}{,} \PYG{l+m+mi}{8}\PYG{p}{,} \PYG{l+m+mi}{8}\PYG{p}{,} \PYG{l+m+mi}{13}\PYG{p}{,} \PYG{l+m+mi}{15}\PYG{p}{,} \PYG{l+m+mi}{18}\PYG{p}{,} \PYG{l+m+mi}{19}\PYG{p}{]}\PYG{p}{,}
       \PYG{p}{[}\PYG{l+m+mi}{7}\PYG{p}{,} \PYG{l+m+mi}{4}\PYG{p}{,} \PYG{l+m+mi}{10}\PYG{p}{,} \PYG{l+m+mi}{9}\PYG{p}{,} \PYG{l+m+mi}{16}\PYG{p}{,} \PYG{l+m+mi}{18}\PYG{p}{,} \PYG{l+m+mi}{20}\PYG{p}{,} \PYG{l+m+mi}{19}\PYG{p}{]}\PYG{p}{,}
       \PYG{p}{[}\PYG{l+m+mi}{9}\PYG{p}{,} \PYG{l+m+mi}{10}\PYG{p}{,} \PYG{l+m+mi}{5}\PYG{p}{,} \PYG{l+m+mi}{27}\PYG{p}{,} \PYG{l+m+mi}{23}\PYG{p}{,} \PYG{l+m+mi}{21}\PYG{p}{,} \PYG{l+m+mi}{21}\PYG{p}{,} \PYG{l+m+mi}{21}\PYG{p}{]}\PYG{p}{,}
       \PYG{p}{[}\PYG{l+m+mi}{11}\PYG{p}{,} \PYG{l+m+mi}{14}\PYG{p}{,} \PYG{l+m+mi}{17}\PYG{p}{,} \PYG{l+m+mi}{26}\PYG{p}{,} \PYG{l+m+mi}{25}\PYG{p}{,} \PYG{l+m+mi}{24}\PYG{p}{,} \PYG{l+m+mi}{23}\PYG{p}{,} \PYG{l+m+mi}{22}\PYG{p}{]}\PYG{p}{]}

\PYG{n}{fig} \PYG{o}{=} \PYG{n}{make\PYGZus{}subplots}\PYG{p}{(}\PYG{n}{rows}\PYG{o}{=}\PYG{l+m+mi}{1}\PYG{p}{,} \PYG{n}{cols}\PYG{o}{=}\PYG{l+m+mi}{2}\PYG{p}{,}
                    \PYG{n}{subplot\PYGZus{}titles}\PYG{o}{=}\PYG{p}{(}\PYG{l+s+s1}{\PYGZsq{}}\PYG{l+s+s1}{Without Smoothing}\PYG{l+s+s1}{\PYGZsq{}}\PYG{p}{,} \PYG{l+s+s1}{\PYGZsq{}}\PYG{l+s+s1}{With Smoothing}\PYG{l+s+s1}{\PYGZsq{}}\PYG{p}{)}\PYG{p}{)}

\PYG{n}{fig}\PYG{o}{.}\PYG{n}{add\PYGZus{}trace}\PYG{p}{(}\PYG{n}{go}\PYG{o}{.}\PYG{n}{Contour}\PYG{p}{(}\PYG{n}{z}\PYG{o}{=}\PYG{n}{z}\PYG{p}{,} \PYG{n}{line\PYGZus{}smoothing}\PYG{o}{=}\PYG{l+m+mi}{0}\PYG{p}{)}\PYG{p}{,} \PYG{l+m+mi}{1}\PYG{p}{,} \PYG{l+m+mi}{1}\PYG{p}{)}
\PYG{n}{fig}\PYG{o}{.}\PYG{n}{add\PYGZus{}trace}\PYG{p}{(}\PYG{n}{go}\PYG{o}{.}\PYG{n}{Contour}\PYG{p}{(}\PYG{n}{z}\PYG{o}{=}\PYG{n}{z}\PYG{p}{,} \PYG{n}{line\PYGZus{}smoothing}\PYG{o}{=}\PYG{l+m+mf}{0.85}\PYG{p}{)}\PYG{p}{,} \PYG{l+m+mi}{1}\PYG{p}{,} \PYG{l+m+mi}{2}\PYG{p}{)}

\PYG{n}{fig}\PYG{o}{.}\PYG{n}{show}\PYG{p}{(}\PYG{p}{)}
\end{sphinxVerbatim}

\end{sphinxuseclass}\end{sphinxVerbatimInput}
\begin{sphinxVerbatimOutput}

\begin{sphinxuseclass}{cell_output}
\end{sphinxuseclass}\end{sphinxVerbatimOutput}

\end{sphinxuseclass}
\begin{sphinxuseclass}{cell}\begin{sphinxVerbatimInput}

\begin{sphinxuseclass}{cell_input}
\begin{sphinxVerbatim}[commandchars=\\\{\}]
\PYG{o}{\PYGZpc{}}\PYG{k}{load\PYGZus{}ext} tikz\PYGZus{}magic
\end{sphinxVerbatim}

\end{sphinxuseclass}\end{sphinxVerbatimInput}

\end{sphinxuseclass}
\begin{sphinxuseclass}{cell}\begin{sphinxVerbatimInput}

\begin{sphinxuseclass}{cell_input}
\begin{sphinxVerbatim}[commandchars=\\\{\}]
\PYG{o}{\PYGZpc{}\PYGZpc{}}\PYG{k}{tikz}
\PYGZbs{}begin\PYGZob{}tikzpicture\PYGZcb{}
\PYGZbs{}begin\PYGZob{}axis\PYGZcb{}[xmax=9,ymax=9, samples=50]
  \PYGZbs{}addplot[blue, ultra thick] (x,x*x);
  \PYGZbs{}addplot[red,  ultra thick] (x*x,x);
\PYGZbs{}end\PYGZob{}axis\PYGZcb{}
\PYGZbs{}end\PYGZob{}tikzpicture\PYGZcb{}
\end{sphinxVerbatim}

\end{sphinxuseclass}\end{sphinxVerbatimInput}
\begin{sphinxVerbatimOutput}

\begin{sphinxuseclass}{cell_output}
\begin{sphinxVerbatim}[commandchars=\\\{\}]
This is XeTeX, Version 3.14159265\PYGZhy{}2.6\PYGZhy{}0.99999 (TeX Live 2018) (preloaded format=xelatex)
 restricted \PYGZbs{}write18 enabled.
entering extended mode
(/var/folders/c4/wb13541d4n104gj2c8s0pdf40000gn/T/tmplp9iwhdk/tikzfile.tex
LaTeX2e \PYGZlt{}2018\PYGZhy{}04\PYGZhy{}01\PYGZgt{} patch level 2
Babel \PYGZlt{}3.18\PYGZgt{} and hyphenation patterns for 84 language(s) loaded.
(/usr/local/texlive/2018/texmf\PYGZhy{}dist/tex/latex/standalone/standalone.cls
Document Class: standalone 2018/03/26 v1.3a Class to compile TeX sub\PYGZhy{}files stan
dalone
(/usr/local/texlive/2018/texmf\PYGZhy{}dist/tex/latex/tools/shellesc.sty)
(/usr/local/texlive/2018/texmf\PYGZhy{}dist/tex/generic/oberdiek/ifluatex.sty)
(/usr/local/texlive/2018/texmf\PYGZhy{}dist/tex/generic/oberdiek/ifpdf.sty)
(/usr/local/texlive/2018/texmf\PYGZhy{}dist/tex/generic/ifxetex/ifxetex.sty)
(/usr/local/texlive/2018/texmf\PYGZhy{}dist/tex/latex/xkeyval/xkeyval.sty
(/usr/local/texlive/2018/texmf\PYGZhy{}dist/tex/generic/xkeyval/xkeyval.tex
(/usr/local/texlive/2018/texmf\PYGZhy{}dist/tex/generic/xkeyval/xkvutils.tex
(/usr/local/texlive/2018/texmf\PYGZhy{}dist/tex/generic/xkeyval/keyval.tex))))
(/usr/local/texlive/2018/texmf\PYGZhy{}dist/tex/latex/standalone/standalone.cfg)
(/usr/local/texlive/2018/texmf\PYGZhy{}dist/tex/latex/base/article.cls
Document Class: article 2014/09/29 v1.4h Standard LaTeX document class
(/usr/local/texlive/2018/texmf\PYGZhy{}dist/tex/latex/base/size10.clo))
(/usr/local/texlive/2018/texmf\PYGZhy{}dist/tex/latex/pgf/frontendlayer/tikz.sty
(/usr/local/texlive/2018/texmf\PYGZhy{}dist/tex/latex/pgf/basiclayer/pgf.sty
(/usr/local/texlive/2018/texmf\PYGZhy{}dist/tex/latex/pgf/utilities/pgfrcs.sty
(/usr/local/texlive/2018/texmf\PYGZhy{}dist/tex/generic/pgf/utilities/pgfutil\PYGZhy{}common.te
x
(/usr/local/texlive/2018/texmf\PYGZhy{}dist/tex/generic/pgf/utilities/pgfutil\PYGZhy{}common\PYGZhy{}li
sts.tex))
(/usr/local/texlive/2018/texmf\PYGZhy{}dist/tex/generic/pgf/utilities/pgfutil\PYGZhy{}latex.def
(/usr/local/texlive/2018/texmf\PYGZhy{}dist/tex/latex/ms/everyshi.sty))
(/usr/local/texlive/2018/texmf\PYGZhy{}dist/tex/generic/pgf/utilities/pgfrcs.code.tex))
(/usr/local/texlive/2018/texmf\PYGZhy{}dist/tex/latex/pgf/basiclayer/pgfcore.sty
(/usr/local/texlive/2018/texmf\PYGZhy{}dist/tex/latex/graphics/graphicx.sty
(/usr/local/texlive/2018/texmf\PYGZhy{}dist/tex/latex/graphics/graphics.sty
(/usr/local/texlive/2018/texmf\PYGZhy{}dist/tex/latex/graphics/trig.sty)
(/usr/local/texlive/2018/texmf\PYGZhy{}dist/tex/latex/graphics\PYGZhy{}cfg/graphics.cfg)
(/usr/local/texlive/2018/texmf\PYGZhy{}dist/tex/latex/graphics\PYGZhy{}def/xetex.def)))
(/usr/local/texlive/2018/texmf\PYGZhy{}dist/tex/latex/pgf/systemlayer/pgfsys.sty
(/usr/local/texlive/2018/texmf\PYGZhy{}dist/tex/generic/pgf/systemlayer/pgfsys.code.tex
(/usr/local/texlive/2018/texmf\PYGZhy{}dist/tex/generic/pgf/utilities/pgfkeys.code.tex
(/usr/local/texlive/2018/texmf\PYGZhy{}dist/tex/generic/pgf/utilities/pgfkeysfiltered.c
ode.tex))
(/usr/local/texlive/2018/texmf\PYGZhy{}dist/tex/generic/pgf/systemlayer/pgf.cfg)
(/usr/local/texlive/2018/texmf\PYGZhy{}dist/tex/generic/pgf/systemlayer/pgfsys\PYGZhy{}xetex.de
f
(/usr/local/texlive/2018/texmf\PYGZhy{}dist/tex/generic/pgf/systemlayer/pgfsys\PYGZhy{}dvipdfmx
.def
(/usr/local/texlive/2018/texmf\PYGZhy{}dist/tex/generic/pgf/systemlayer/pgfsys\PYGZhy{}common\PYGZhy{}p
df.def))))
(/usr/local/texlive/2018/texmf\PYGZhy{}dist/tex/generic/pgf/systemlayer/pgfsyssoftpath.
code.tex)
(/usr/local/texlive/2018/texmf\PYGZhy{}dist/tex/generic/pgf/systemlayer/pgfsysprotocol.
code.tex)) (/usr/local/texlive/2018/texmf\PYGZhy{}dist/tex/latex/xcolor/xcolor.sty
(/usr/local/texlive/2018/texmf\PYGZhy{}dist/tex/latex/graphics\PYGZhy{}cfg/color.cfg))
(/usr/local/texlive/2018/texmf\PYGZhy{}dist/tex/generic/pgf/basiclayer/pgfcore.code.tex
(/usr/local/texlive/2018/texmf\PYGZhy{}dist/tex/generic/pgf/math/pgfmath.code.tex
(/usr/local/texlive/2018/texmf\PYGZhy{}dist/tex/generic/pgf/math/pgfmathcalc.code.tex
(/usr/local/texlive/2018/texmf\PYGZhy{}dist/tex/generic/pgf/math/pgfmathutil.code.tex)
(/usr/local/texlive/2018/texmf\PYGZhy{}dist/tex/generic/pgf/math/pgfmathparser.code.tex
\end{sphinxVerbatim}

\begin{sphinxVerbatim}[commandchars=\\\{\}]
)
(/usr/local/texlive/2018/texmf\PYGZhy{}dist/tex/generic/pgf/math/pgfmathfunctions.code.
tex
(/usr/local/texlive/2018/texmf\PYGZhy{}dist/tex/generic/pgf/math/pgfmathfunctions.basic
.code.tex)
(/usr/local/texlive/2018/texmf\PYGZhy{}dist/tex/generic/pgf/math/pgfmathfunctions.trigo
nometric.code.tex)
(/usr/local/texlive/2018/texmf\PYGZhy{}dist/tex/generic/pgf/math/pgfmathfunctions.rando
m.code.tex)
(/usr/local/texlive/2018/texmf\PYGZhy{}dist/tex/generic/pgf/math/pgfmathfunctions.compa
rison.code.tex)
(/usr/local/texlive/2018/texmf\PYGZhy{}dist/tex/generic/pgf/math/pgfmathfunctions.base.
code.tex)
(/usr/local/texlive/2018/texmf\PYGZhy{}dist/tex/generic/pgf/math/pgfmathfunctions.round
.code.tex)
(/usr/local/texlive/2018/texmf\PYGZhy{}dist/tex/generic/pgf/math/pgfmathfunctions.misc.
code.tex)
(/usr/local/texlive/2018/texmf\PYGZhy{}dist/tex/generic/pgf/math/pgfmathfunctions.integ
erarithmetics.code.tex)))
(/usr/local/texlive/2018/texmf\PYGZhy{}dist/tex/generic/pgf/math/pgfmathfloat.code.tex)
)
(/usr/local/texlive/2018/texmf\PYGZhy{}dist/tex/generic/pgf/basiclayer/pgfcorepoints.co
de.tex)
(/usr/local/texlive/2018/texmf\PYGZhy{}dist/tex/generic/pgf/basiclayer/pgfcorepathconst
ruct.code.tex)
(/usr/local/texlive/2018/texmf\PYGZhy{}dist/tex/generic/pgf/basiclayer/pgfcorepathusage
.code.tex)
(/usr/local/texlive/2018/texmf\PYGZhy{}dist/tex/generic/pgf/basiclayer/pgfcorescopes.co
de.tex)
(/usr/local/texlive/2018/texmf\PYGZhy{}dist/tex/generic/pgf/basiclayer/pgfcoregraphicst
ate.code.tex)
(/usr/local/texlive/2018/texmf\PYGZhy{}dist/tex/generic/pgf/basiclayer/pgfcoretransform
ations.code.tex)
(/usr/local/texlive/2018/texmf\PYGZhy{}dist/tex/generic/pgf/basiclayer/pgfcorequick.cod
e.tex)
(/usr/local/texlive/2018/texmf\PYGZhy{}dist/tex/generic/pgf/basiclayer/pgfcoreobjects.c
ode.tex)
(/usr/local/texlive/2018/texmf\PYGZhy{}dist/tex/generic/pgf/basiclayer/pgfcorepathproce
ssing.code.tex)
(/usr/local/texlive/2018/texmf\PYGZhy{}dist/tex/generic/pgf/basiclayer/pgfcorearrows.co
de.tex)
(/usr/local/texlive/2018/texmf\PYGZhy{}dist/tex/generic/pgf/basiclayer/pgfcoreshade.cod
e.tex)
(/usr/local/texlive/2018/texmf\PYGZhy{}dist/tex/generic/pgf/basiclayer/pgfcoreimage.cod
e.tex
(/usr/local/texlive/2018/texmf\PYGZhy{}dist/tex/generic/pgf/basiclayer/pgfcoreexternal.
code.tex))
(/usr/local/texlive/2018/texmf\PYGZhy{}dist/tex/generic/pgf/basiclayer/pgfcorelayers.co
de.tex)
(/usr/local/texlive/2018/texmf\PYGZhy{}dist/tex/generic/pgf/basiclayer/pgfcoretranspare
ncy.code.tex)
(/usr/local/texlive/2018/texmf\PYGZhy{}dist/tex/generic/pgf/basiclayer/pgfcorepatterns.
code.tex)))
(/usr/local/texlive/2018/texmf\PYGZhy{}dist/tex/generic/pgf/modules/pgfmoduleshapes.cod
e.tex)
(/usr/local/texlive/2018/texmf\PYGZhy{}dist/tex/generic/pgf/modules/pgfmoduleplot.code.
tex)
(/usr/local/texlive/2018/texmf\PYGZhy{}dist/tex/latex/pgf/compatibility/pgfcomp\PYGZhy{}version
\PYGZhy{}0\PYGZhy{}65.sty)
(/usr/local/texlive/2018/texmf\PYGZhy{}dist/tex/latex/pgf/compatibility/pgfcomp\PYGZhy{}version
\PYGZhy{}1\PYGZhy{}18.sty))
(/usr/local/texlive/2018/texmf\PYGZhy{}dist/tex/latex/pgf/utilities/pgffor.sty
(/usr/local/texlive/2018/texmf\PYGZhy{}dist/tex/latex/pgf/utilities/pgfkeys.sty
(/usr/local/texlive/2018/texmf\PYGZhy{}dist/tex/generic/pgf/utilities/pgfkeys.code.tex)
) (/usr/local/texlive/2018/texmf\PYGZhy{}dist/tex/latex/pgf/math/pgfmath.sty
(/usr/local/texlive/2018/texmf\PYGZhy{}dist/tex/generic/pgf/math/pgfmath.code.tex))
(/usr/local/texlive/2018/texmf\PYGZhy{}dist/tex/generic/pgf/utilities/pgffor.code.tex
(/usr/local/texlive/2018/texmf\PYGZhy{}dist/tex/generic/pgf/math/pgfmath.code.tex)))
(/usr/local/texlive/2018/texmf\PYGZhy{}dist/tex/generic/pgf/frontendlayer/tikz/tikz.cod
e.tex
(/usr/local/texlive/2018/texmf\PYGZhy{}dist/tex/generic/pgf/libraries/pgflibraryplothan
dlers.code.tex)
(/usr/local/texlive/2018/texmf\PYGZhy{}dist/tex/generic/pgf/modules/pgfmodulematrix.cod
e.tex)
(/usr/local/texlive/2018/texmf\PYGZhy{}dist/tex/generic/pgf/frontendlayer/tikz/librarie
s/tikzlibrarytopaths.code.tex))))
No file tikzfile.aux.
ABD: EveryShipout initializing macros

! LaTeX Error: Environment axis undefined.

See the LaTeX manual or LaTeX Companion for explanation.
Type  H \PYGZlt{}return\PYGZgt{}  for immediate help.
 ...                                              
                                                  
l.8 \PYGZbs{}begin\PYGZob{}axis\PYGZcb{}
                [xmax=9,ymax=9, samples=50]
? 
! Emergency stop.
 ...                                              
                                                  
l.8 \PYGZbs{}begin\PYGZob{}axis\PYGZcb{}
                [xmax=9,ymax=9, samples=50]
No pages of output.
Transcript written on /var/folders/c4/wb13541d4n104gj2c8s0pdf40000gn/T/tmplp9iw
hdk/tikzfile.log.
\end{sphinxVerbatim}

\begin{sphinxVerbatim}[commandchars=\\\{\}]
\PYG{g+gt}{\PYGZhy{}\PYGZhy{}\PYGZhy{}\PYGZhy{}\PYGZhy{}\PYGZhy{}\PYGZhy{}\PYGZhy{}\PYGZhy{}\PYGZhy{}\PYGZhy{}\PYGZhy{}\PYGZhy{}\PYGZhy{}\PYGZhy{}\PYGZhy{}\PYGZhy{}\PYGZhy{}\PYGZhy{}\PYGZhy{}\PYGZhy{}\PYGZhy{}\PYGZhy{}\PYGZhy{}\PYGZhy{}\PYGZhy{}\PYGZhy{}\PYGZhy{}\PYGZhy{}\PYGZhy{}\PYGZhy{}\PYGZhy{}\PYGZhy{}\PYGZhy{}\PYGZhy{}\PYGZhy{}\PYGZhy{}\PYGZhy{}\PYGZhy{}\PYGZhy{}\PYGZhy{}\PYGZhy{}\PYGZhy{}\PYGZhy{}\PYGZhy{}\PYGZhy{}\PYGZhy{}\PYGZhy{}\PYGZhy{}\PYGZhy{}\PYGZhy{}\PYGZhy{}\PYGZhy{}\PYGZhy{}\PYGZhy{}\PYGZhy{}\PYGZhy{}\PYGZhy{}\PYGZhy{}\PYGZhy{}\PYGZhy{}\PYGZhy{}\PYGZhy{}\PYGZhy{}\PYGZhy{}\PYGZhy{}\PYGZhy{}\PYGZhy{}\PYGZhy{}\PYGZhy{}\PYGZhy{}\PYGZhy{}\PYGZhy{}\PYGZhy{}\PYGZhy{}}
\PYG{n+ne}{Exception}\PYG{g+gWhitespace}{                                 }Traceback (most recent call last)
\PYG{o}{/}\PYG{n}{var}\PYG{o}{/}\PYG{n}{folders}\PYG{o}{/}\PYG{n}{c4}\PYG{o}{/}\PYG{n}{wb13541d4n104gj2c8s0pdf40000gn}\PYG{o}{/}\PYG{n}{T}\PYG{o}{/}\PYG{n}{ipykernel\PYGZus{}47988}\PYG{o}{/}\PYG{l+m+mf}{2552934825.}\PYG{n}{py} \PYG{o+ow}{in} \PYG{o}{\PYGZlt{}}\PYG{n}{module}\PYG{o}{\PYGZgt{}}
\PYG{n+ne}{\PYGZhy{}\PYGZhy{}\PYGZhy{}\PYGZhy{}\PYGZgt{} }\PYG{l+m+mi}{1} \PYG{n}{get\PYGZus{}ipython}\PYG{p}{(}\PYG{p}{)}\PYG{o}{.}\PYG{n}{run\PYGZus{}cell\PYGZus{}magic}\PYG{p}{(}\PYG{l+s+s1}{\PYGZsq{}}\PYG{l+s+s1}{tikz}\PYG{l+s+s1}{\PYGZsq{}}\PYG{p}{,} \PYG{l+s+s1}{\PYGZsq{}}\PYG{l+s+s1}{\PYGZsq{}}\PYG{p}{,} \PYG{l+s+s1}{\PYGZsq{}}\PYG{l+s+se}{\PYGZbs{}\PYGZbs{}}\PYG{l+s+s1}{begin}\PYG{l+s+si}{\PYGZob{}tikzpicture\PYGZcb{}}\PYG{l+s+se}{\PYGZbs{}n}\PYG{l+s+se}{\PYGZbs{}\PYGZbs{}}\PYG{l+s+s1}{begin}\PYG{l+s+si}{\PYGZob{}axis\PYGZcb{}}\PYG{l+s+s1}{[xmax=9,ymax=9, samples=50]}\PYG{l+s+se}{\PYGZbs{}n}\PYG{l+s+s1}{  }\PYG{l+s+se}{\PYGZbs{}\PYGZbs{}}\PYG{l+s+s1}{addplot[blue, ultra thick] (x,x*x);}\PYG{l+s+se}{\PYGZbs{}n}\PYG{l+s+s1}{  }\PYG{l+s+se}{\PYGZbs{}\PYGZbs{}}\PYG{l+s+s1}{addplot[red,  ultra thick] (x*x,x);}\PYG{l+s+se}{\PYGZbs{}n}\PYG{l+s+se}{\PYGZbs{}\PYGZbs{}}\PYG{l+s+s1}{end}\PYG{l+s+si}{\PYGZob{}axis\PYGZcb{}}\PYG{l+s+se}{\PYGZbs{}n}\PYG{l+s+se}{\PYGZbs{}\PYGZbs{}}\PYG{l+s+s1}{end}\PYG{l+s+si}{\PYGZob{}tikzpicture\PYGZcb{}}\PYG{l+s+se}{\PYGZbs{}n}\PYG{l+s+s1}{\PYGZsq{}}\PYG{p}{)}

\PYG{n+nn}{\PYGZti{}/opt/anaconda3/lib/python3.8/site\PYGZhy{}packages/IPython/core/interactiveshell.py} in \PYG{n+ni}{run\PYGZus{}cell\PYGZus{}magic}\PYG{n+nt}{(self, magic\PYGZus{}name, line, cell)}
\PYG{g+gWhitespace}{   }\PYG{l+m+mi}{2417}             \PYG{k}{with} \PYG{n+nb+bp}{self}\PYG{o}{.}\PYG{n}{builtin\PYGZus{}trap}\PYG{p}{:}
\PYG{g+gWhitespace}{   }\PYG{l+m+mi}{2418}                 \PYG{n}{args} \PYG{o}{=} \PYG{p}{(}\PYG{n}{magic\PYGZus{}arg\PYGZus{}s}\PYG{p}{,} \PYG{n}{cell}\PYG{p}{)}
\PYG{n+ne}{\PYGZhy{}\PYGZgt{} }\PYG{l+m+mi}{2419}                 \PYG{n}{result} \PYG{o}{=} \PYG{n}{fn}\PYG{p}{(}\PYG{o}{*}\PYG{n}{args}\PYG{p}{,} \PYG{o}{*}\PYG{o}{*}\PYG{n}{kwargs}\PYG{p}{)}
\PYG{g+gWhitespace}{   }\PYG{l+m+mi}{2420}             \PYG{k}{return} \PYG{n}{result}
\PYG{g+gWhitespace}{   }\PYG{l+m+mi}{2421} 

\PYG{n+nn}{\PYGZti{}/opt/anaconda3/lib/python3.8/site\PYGZhy{}packages/decorator.py} in \PYG{n+ni}{fun}\PYG{n+nt}{(*args, **kw)}
\PYG{g+gWhitespace}{    }\PYG{l+m+mi}{230}             \PYG{k}{if} \PYG{o+ow}{not} \PYG{n}{kwsyntax}\PYG{p}{:}
\PYG{g+gWhitespace}{    }\PYG{l+m+mi}{231}                 \PYG{n}{args}\PYG{p}{,} \PYG{n}{kw} \PYG{o}{=} \PYG{n}{fix}\PYG{p}{(}\PYG{n}{args}\PYG{p}{,} \PYG{n}{kw}\PYG{p}{,} \PYG{n}{sig}\PYG{p}{)}
\PYG{n+ne}{\PYGZhy{}\PYGZhy{}\PYGZgt{} }\PYG{l+m+mi}{232}             \PYG{k}{return} \PYG{n}{caller}\PYG{p}{(}\PYG{n}{func}\PYG{p}{,} \PYG{o}{*}\PYG{p}{(}\PYG{n}{extras} \PYG{o}{+} \PYG{n}{args}\PYG{p}{)}\PYG{p}{,} \PYG{o}{*}\PYG{o}{*}\PYG{n}{kw}\PYG{p}{)}
\PYG{g+gWhitespace}{    }\PYG{l+m+mi}{233}     \PYG{n}{fun}\PYG{o}{.}\PYG{n+nv+vm}{\PYGZus{}\PYGZus{}name\PYGZus{}\PYGZus{}} \PYG{o}{=} \PYG{n}{func}\PYG{o}{.}\PYG{n+nv+vm}{\PYGZus{}\PYGZus{}name\PYGZus{}\PYGZus{}}
\PYG{g+gWhitespace}{    }\PYG{l+m+mi}{234}     \PYG{n}{fun}\PYG{o}{.}\PYG{n+nv+vm}{\PYGZus{}\PYGZus{}doc\PYGZus{}\PYGZus{}} \PYG{o}{=} \PYG{n}{func}\PYG{o}{.}\PYG{n+nv+vm}{\PYGZus{}\PYGZus{}doc\PYGZus{}\PYGZus{}}

\PYG{n+nn}{\PYGZti{}/opt/anaconda3/lib/python3.8/site\PYGZhy{}packages/IPython/core/magic.py} in \PYG{n+ni}{\PYGZlt{}lambda\PYGZgt{}}\PYG{n+nt}{(f, *a, **k)}
\PYG{g+gWhitespace}{    }\PYG{l+m+mi}{185}     \PYG{c+c1}{\PYGZsh{} but it\PYGZsq{}s overkill for just that one bit of state.}
\PYG{g+gWhitespace}{    }\PYG{l+m+mi}{186}     \PYG{k}{def} \PYG{n+nf}{magic\PYGZus{}deco}\PYG{p}{(}\PYG{n}{arg}\PYG{p}{)}\PYG{p}{:}
\PYG{n+ne}{\PYGZhy{}\PYGZhy{}\PYGZgt{} }\PYG{l+m+mi}{187}         \PYG{n}{call} \PYG{o}{=} \PYG{k}{lambda} \PYG{n}{f}\PYG{p}{,} \PYG{o}{*}\PYG{n}{a}\PYG{p}{,} \PYG{o}{*}\PYG{o}{*}\PYG{n}{k}\PYG{p}{:} \PYG{n}{f}\PYG{p}{(}\PYG{o}{*}\PYG{n}{a}\PYG{p}{,} \PYG{o}{*}\PYG{o}{*}\PYG{n}{k}\PYG{p}{)}
\PYG{g+gWhitespace}{    }\PYG{l+m+mi}{188} 
\PYG{g+gWhitespace}{    }\PYG{l+m+mi}{189}         \PYG{k}{if} \PYG{n}{callable}\PYG{p}{(}\PYG{n}{arg}\PYG{p}{)}\PYG{p}{:}

\PYG{n+nn}{\PYGZti{}/opt/anaconda3/lib/python3.8/site\PYGZhy{}packages/tikz\PYGZus{}magic/tikz.py} in \PYG{n+ni}{tikz}\PYG{n+nt}{(self, line, cell)}
\PYG{g+gWhitespace}{     }\PYG{l+m+mi}{75} 
\PYG{g+gWhitespace}{     }\PYG{l+m+mi}{76}         \PYG{c+c1}{\PYGZsh{} compile and convert, returning Image data}
\PYG{n+ne}{\PYGZhy{}\PYGZhy{}\PYGZhy{}\PYGZgt{} }\PYG{l+m+mi}{77}         \PYG{k}{return} \PYG{n}{latex2image}\PYG{p}{(}\PYG{n}{latex}\PYG{p}{,} \PYG{n+nb}{int}\PYG{p}{(}\PYG{n}{args}\PYG{o}{.}\PYG{n}{scale}\PYG{o}{*}\PYG{l+m+mi}{300}\PYG{p}{)}\PYG{p}{,} \PYG{n}{args}\PYG{o}{.}\PYG{n}{export\PYGZus{}file}\PYG{p}{)}
\PYG{g+gWhitespace}{     }\PYG{l+m+mi}{78} 
\PYG{g+gWhitespace}{     }\PYG{l+m+mi}{79} \PYG{k}{def} \PYG{n+nf}{latex2image}\PYG{p}{(}\PYG{n}{latex}\PYG{p}{,} \PYG{n}{density}\PYG{p}{,} \PYG{n}{export\PYGZus{}file}\PYG{o}{=}\PYG{k+kc}{None}\PYG{p}{)}\PYG{p}{:}

\PYG{n+nn}{\PYGZti{}/opt/anaconda3/lib/python3.8/site\PYGZhy{}packages/tikz\PYGZus{}magic/tikz.py} in \PYG{n+ni}{latex2image}\PYG{n+nt}{(latex, density, export\PYGZus{}file)}
\PYG{g+gWhitespace}{     }\PYG{l+m+mi}{91} 
\PYG{g+gWhitespace}{     }\PYG{l+m+mi}{92}         \PYG{k}{if} \PYG{o+ow}{not} \PYG{n}{isfile}\PYG{p}{(}\PYG{n}{temp\PYGZus{}pdf}\PYG{p}{)}\PYG{p}{:}
\PYG{n+ne}{\PYGZhy{}\PYGZhy{}\PYGZhy{}\PYGZgt{} }\PYG{l+m+mi}{93}             \PYG{k}{raise} \PYG{n+ne}{Exception}\PYG{p}{(}\PYG{l+s+s1}{\PYGZsq{}}\PYG{l+s+s1}{pdflatex did not produce a PDF file.}\PYG{l+s+s1}{\PYGZsq{}}\PYG{p}{)}
\PYG{g+gWhitespace}{     }\PYG{l+m+mi}{94} 
\PYG{g+gWhitespace}{     }\PYG{l+m+mi}{95}         \PYG{k}{if} \PYG{n}{export\PYGZus{}file}\PYG{p}{:}

\PYG{n+ne}{Exception}: pdflatex did not produce a PDF file.
\end{sphinxVerbatim}

\end{sphinxuseclass}\end{sphinxVerbatimOutput}

\end{sphinxuseclass}
\sphinxstepscope


\part{Statistique inférentielle}

\sphinxstepscope


\chapter{Introduction}
\label{\detokenize{Statistique_inf_xe9rentielle/introduction:introduction}}\label{\detokenize{Statistique_inf_xe9rentielle/introduction::doc}}
\sphinxstepscope


\chapter{Estimation de paramètres}
\label{\detokenize{Statistique_inf_xe9rentielle/estimation_de_parametres:estimation-de-parametres}}\label{\detokenize{Statistique_inf_xe9rentielle/estimation_de_parametres::doc}}
\sphinxstepscope


\chapter{Propriétés d’un estimateur}
\label{\detokenize{Statistique_inf_xe9rentielle/proprietes_estimateur:proprietes-dun-estimateur}}\label{\detokenize{Statistique_inf_xe9rentielle/proprietes_estimateur::doc}}
\sphinxstepscope


\chapter{Estimation par intervalle}
\label{\detokenize{Statistique_inf_xe9rentielle/estimation_par_intervalle:estimation-par-intervalle}}\label{\detokenize{Statistique_inf_xe9rentielle/estimation_par_intervalle::doc}}
\sphinxstepscope


\chapter{Tests d’hypothèses statistiques}
\label{\detokenize{Statistique_inf_xe9rentielle/tests_hypotheses_statistiques:tests-dhypotheses-statistiques}}\label{\detokenize{Statistique_inf_xe9rentielle/tests_hypotheses_statistiques::doc}}
\sphinxstepscope


\chapter{Tests et IC}
\label{\detokenize{Statistique_inf_xe9rentielle/tests_et_ic:tests-et-ic}}\label{\detokenize{Statistique_inf_xe9rentielle/tests_et_ic::doc}}
\sphinxstepscope


\chapter{Test du chi2}
\label{\detokenize{Statistique_inf_xe9rentielle/test_chi2:test-du-chi2}}\label{\detokenize{Statistique_inf_xe9rentielle/test_chi2::doc}}
\sphinxstepscope


\chapter{Puissance d’un test}
\label{\detokenize{Statistique_inf_xe9rentielle/puissance_test:puissance-dun-test}}\label{\detokenize{Statistique_inf_xe9rentielle/puissance_test::doc}}
\sphinxstepscope


\part{Régréssion linéaire}

\sphinxstepscope


\chapter{Introduction}
\label{\detokenize{R_xe9gr_xe9ssion_lin_xe9aire/introduction:introduction}}\label{\detokenize{R_xe9gr_xe9ssion_lin_xe9aire/introduction::doc}}
\sphinxstepscope


\chapter{Régréssion linéaire: cas général}
\label{\detokenize{R_xe9gr_xe9ssion_lin_xe9aire/cas_general:regression-lineaire-cas-general}}\label{\detokenize{R_xe9gr_xe9ssion_lin_xe9aire/cas_general::doc}}
\sphinxstepscope


\chapter{Régréssion linéaire: tests}
\label{\detokenize{R_xe9gr_xe9ssion_lin_xe9aire/tests:regression-lineaire-tests}}\label{\detokenize{R_xe9gr_xe9ssion_lin_xe9aire/tests::doc}}
\sphinxstepscope


\chapter{Régréssion linéaire: hypothèses et diagnostics}
\label{\detokenize{R_xe9gr_xe9ssion_lin_xe9aire/hypotheses_et_diagnostics:regression-lineaire-hypotheses-et-diagnostics}}\label{\detokenize{R_xe9gr_xe9ssion_lin_xe9aire/hypotheses_et_diagnostics::doc}}





\renewcommand{\indexname}{Proof Index}
\begin{sphinxtheindex}
\let\bigletter\sphinxstyleindexlettergroup
\bigletter{theorem\sphinxhyphen{}0}
\item\relax\sphinxstyleindexentry{theorem\sphinxhyphen{}0}\sphinxstyleindexextra{Probabilité/arrangements\_et\_combinaisons}\sphinxstyleindexpageref{Probabilité/arrangements_et_combinaisons:\detokenize{theorem-0}}
\end{sphinxtheindex}

\renewcommand{\indexname}{Index}
\printindex
\end{document}