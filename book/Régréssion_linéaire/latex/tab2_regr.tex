
\documentclass{standalone}

\usepackage{pgf}
\usepackage{pgfpages}


\usepackage{amsmath}
\usepackage{amssymb}



\usepackage{amsthm}
\usepackage{thmtools}

%\usefonttheme{professionalfonts} % using non standard fonts for beamer
%\usepackage{fontspec}
%\setmainfont{Source Sans Pro}

% everyone:
\usepackage[utf8]{inputenc}
\usepackage[T1]{fontenc}
\usepackage{lmodern}




\usepackage{xcolor}
\usepackage{multicol}
\usepackage{wrapfig}
\definecolor{ggorange}{RGB}{237,116,108} % trouvé les couleurs dans colorimètre numérique, avec sRVB
\definecolor{ggblue}{RGB}{88,192,197} 
\newcommand{\dred}[1]{\begin{color}{red}{#1}\end{color}}
\newcommand{\dblue}[1]{\begin{color}{blue}{#1}\end{color}}
\usepackage{tikz}
\usepackage{pgfplots}
\usetikzlibrary{arrows,backgrounds,decorations}
\usepackage{tikzsymbols}

\usepackage{environ} % useful for the "scaletikzpicturetowidth" commands



\newcommand{\target}[1]{%
	\foreach \r/\col in {2.5/blue!10!white, 2/blue!20!white, 1.5/white, 1/red!10!white, 0.5/red!30!white, 0.05/red!50!white} {
		\draw [
		fill = \col%,
		% fill opacity = 0.15
		] (0, 0) circle (\r cm);
	}
}%

\newcommand{\points}[3]{%
	\foreach \i in {1, 2, ..., 20} {
		\pgfmathsetmacro{\xcoord}{#1 + rand * #3}
		\pgfmathsetmacro{\ycoord}{#2 + rand * #3}
		\draw[fill = red] (\xcoord, \ycoord) circle (2pt);
	}
}%


\newcommand{\circlevar}[3]{%
	\draw[color = green] (#1, #2) circle (#3 cm);
	\node[text = green] at (#1, #2) {$\bullet$};
}%

\def\border{
	\draw (-2.65, -2.75) -- ++(5.25, 0);
	\draw (2.75, 2.65) -- ++(0, -5.25);
}
\usepackage{3dplot}


\begin{document}
\tdplotsetmaincoords{60}{110}

%define polar coordinates for some vector
%TODO: look into using 3d spherical coordinate system
\pgfmathsetmacro{\rvec}{.8}
\pgfmathsetmacro{\thetavec}{30}
\pgfmathsetmacro{\phivec}{60}


	\begin{tikzpicture}[scale=5,tdplot_main_coords]
		
		%set up some coordinates 
		%-----------------------
		\coordinate (O) at (0,0,0);
		
		%determine a coordinate (P) using (r,\theta,\phi) coordinates.  This command
		%also determines (Pxy), (Pxz), and (Pyz): the xy-, xz-, and yz-projections
		%of the point (P).
		%syntax: \tdplotsetcoord{Coordinate name without parentheses}{r}{\theta}{\phi}
		\tdplotsetcoord{P}{\rvec}{\thetavec}{\phivec}
		
		%draw figure contents
		%--------------------
		
		%draw the main coordinate system axes
		\draw[thick,->] (0,0,0) -- (1,0,0) node[anchor=north east]{};
		\draw[thick,->] (0,0,0) -- (0,1,0) node[anchor=north west]{};
		\draw[thick,->] (0,0,0) -- (0,0,1) node[anchor=south]{$\text{span}(\mathbf{X}^{\perp})$};
		
		%draw a vector from origin to point (P) 
		\draw[-stealth,color=black] (O) -- (P)node[right]{$\mathbf{y}$};
		
		%draw projection on xy plane, and a connecting line
		\draw[fill = gray, opacity = 0.3] (0,0,0)--(1,0,0)--(1,1,0)--(0,1,0)--cycle;
		\draw (0.7,0.7, 0) node {$\text{span}(\mathbf{X})$};
		\draw[dashed, color=black] (O) -- (Pxy)node[right]{$\hat{\mathbf{y}}$};
		\draw[dashed, color=black] (P) -- (Pxy)node[midway, right]{$\mathbf{e}$};
	\end{tikzpicture}
\end{document}